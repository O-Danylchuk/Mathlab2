\documentclass{article}
\usepackage[utf8]{inputenc}
\usepackage[T2A]{fontenc}
\usepackage[a4paper, left=0.5in, right=0.5in, top=1in, bottom=0.5in]{geometry}
\usepackage[ukrainian]{babel}
\usepackage{amsmath}

\begin{document}

\section*{Власні значення і власні вектори матриці}

Власні значення і власні вектори матриці використовуються для аналізу властивостей лінійних перетворень. Для квадратної матриці \( A \) власний вектор \( \mathbf{v} \) і власне значення \( \lambda \) визначаються так:

\[ A\mathbf{v} = \lambda\mathbf{v} \text{, де \( \mathbf{v} \) — ненульовий вектор, а \( \lambda \) — скаляр.}\]



\subsection*{Обчислення власних значень і власних векторів}

\begin{enumerate}
   \item \textbf{Знайти власні значення:}
   Розв'язати рівняння:
   \[
   \det(A - \lambda I) = 0
   \]

   \item \textbf{Знайти власні вектори:}
   Для кожного власного значення \( \lambda \) розв'язати систему лінійних рівнянь:
   \[
   (A - \lambda I)\mathbf{v} = 0
   \]
\end{enumerate}

\section*{Властивості власних векторів симетричних матриць}

\begin{enumerate}
   \item \textbf{Дійсність власних значень:} Всі власні значення симетричної матриці є дійсними числами.

   \item \textbf{Ортогональність власних векторів:} Власні вектори, що відповідають різним власним значенням, є ортогональними. Це означає, що \( \mathbf{v}_i^T \mathbf{v}_j = 0 \) для \( i \neq j \).

   \item \textbf{Ортогональність власних векторів:} Власні вектори, що відповідають різним власним значенням, є ортогональними. Це означає, що \( \mathbf{v}_i^T \mathbf{v}_j = 0 \) для \( i \neq j \).

   \item \textbf{Ортонормований базис:} Власні вектори симетричної матриці можуть бути вибрані ортонормованими, тобто утворюють ортонормований базис в просторі.

\end{enumerate}



\section*{Недоліки PCA і стратегії їх подолання}

Principal Component Analysis (PCA) має декілька недоліків:

1. \textbf{Чутливість до масштабів даних:} Якщо ознаки мають різні масштаби, PCA 
може бути некоректним.

   \textbf{Стратегія:} Стандартизація або нормалізація даних перед застосуванням PCA.

2. \textbf{Втрата інтерпретації:} Головні компоненти можуть бути складними 
для інтерпретації.

   \textbf{Стратегія:} Використання обертання компонент (варімакс-обертання) для полегшення інтерпретації.

3. \textbf{Лінійність методу:} PCA виявляє лише лінійні кореляції між ознаками.

   \textbf{Стратегія:} Використання нелінійних варіантів, таких як Kernel PCA.

4. \textbf{Чутливість до шуму:} PCA може бути чутливим до шуму в даних.

   \textbf{Стратегія:} Використання методів попередньої обробки для видалення шуму або зменшення розмірності перед застосуванням PCA.

\section*{Переваги діагоналізації матриці в криптографії}

\begin{enumerate}
   \item \textbf{Прискорення обчислень:} Діагоналізація спрощує піднесення матриці до степеня, що корисно для шифрування та дешифрування.
   Якщо \( A = PDP^{-1} \), то \( A^k = PD^kP^{-1} \), де \( D \) — діагональна матриця, піднесення до степеня якої є тривіальним.

   \item \textbf{Спростування інверсії:} Інверсія діагональної матриці також проста:
   \( D^{-1} \) обчислюється легко, якщо відомі всі ненульові діагональні елементи \( D \).

\end{enumerate}


\subsection*{Застосування в шифруванні та дешифруванні}

\begin{enumerate}
   \item \textbf{Шифрування:} Перетворення повідомлення в матрицю та піднесення цієї матриці до певного степеня.
   \item \textbf{Дешифрування:} Використання оберненої матриці для відновлення оригінального повідомлення.
\end{enumerate}

В деяких схемах шифрування можна використовувати діагоналізацію для підвищення 
ефективності та безпеки перетворень.

\end{document}
